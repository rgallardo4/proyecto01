\documentclass{article}

\usepackage{Sweave}
\begin{document}
\input{probando-concordance}

hola
\begin{Schunk}
\begin{Sinput}
> suma <- function(x,y){
+   x+y  
+   
+   
+ }
> suma(2,5)
\end{Sinput}
\begin{Soutput}
[1] 7
\end{Soutput}
\begin{Sinput}
> suma(5,2)
\end{Sinput}
\begin{Soutput}
[1] 7
\end{Soutput}
\begin{Sinput}
> a <- c(2, 5, -3, -14, 4, 9, 19)
> b <- c(-2, -1, 11, 5, 2, 7, 4)
> c <- c(1, 5)
> suma(a, b)
\end{Sinput}
\begin{Soutput}
[1]  0  4  8 -9  6 16 23
\end{Soutput}
\begin{Sinput}
> suma(a, c)
\end{Sinput}
\begin{Soutput}
[1]  3 10 -2 -9  5 14 20
\end{Soutput}
\begin{Sinput}
> suma <- function(x,y){
+   adicion <- x+y  
+   resta <- x-y
+   print(adicion)
+   print(resta) #Es necesario usar el comando print para imprimir todas las instrucciones
+ }
> 
\end{Sinput}
\end{Schunk}



\end{document}
